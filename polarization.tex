\usepackage{amsmath}
\begin{document}
Let's consider light as a wave represented by a vector who's magnitude is proportional to the intensity of the light. The idea is that the vector points in the direction of oscillation of the wave. Thus, if we are observing a light source and project, consider a plane defined by the vector pointing from the observer to the source (as a normal to the plane), we see that the waves from the source are distributed uniformly around the circle who's radius is the intensity. (Note, it might actually be a variety of intensities, for simplicity we shall assume they are all the same). When introducing a polarizer, it selects a direction and we project each vector onto that direction and that is the intesity of the light that makes it through a polarizer. For the purposes of simplifying the problem, the polarizers will otherwise not obstruct any light. If all the light came through with the correct orientation, there would be no loss of intensity. First, let's consider a toy problem where all rays of light generated by the light source come out at the same intensity and with a polarization that is an angle uniformly selected around the unit circle. To get the total intensity of light generated by this source, we integrate across the probability distribution scaling it based on the value of the intensity of that generated ray:
\begin{gather*}
    I_{tot} = \int I(t)p(t)dt
\end{gather*}
where $p(t)$ is the probability of generating a ray of strength $I(t)$. As there is no polarizer, we do not need to worry about the orientation of the ray. If we assume that the rays are all the same intensity around the circle and that the probability is uniformly distributed $I(t)$ is a constant and $p(t) = 1/2\pi$ (uniform distribution on a domain of length 0 to $2\pi$ and the integral becomes:
\begin{gather*}
    I_{tot} = \int_{0}^{2\pi}\frac{I}{2\pi}dt = \frac{I}{2\pi}t\Big|_{0}^{\2\pi} = I - 0 = I
\end{gather*}
If we then add a polarizer, we have to concern ourselves with the \emph{orientation} of the light vector. Consider a polarizer that is aligned with the $y$-axis. The alignment with the $y$-axis is a simplification, but this can be true for any orientation. Also, observe that by symmetry, we only need to perform the same integral in the first quadrant and multiply the result by 4. Thus, if $\vec{v}$ is the light vector generated by the source:
\begin{gather*}
    \vec{v} = I\cos(t)\vec{\hat{x}} + I\sin(t)\vec{hat{y}}\\
    I(t) = \vec{v} \cdot \vec{\hat{y}} = I\sin(t)\\
    p(t) = \frac{1}{2\pi}
\end{gather*}
where $I = I_{tot}$ is a constant. This is the proportion of the light making it through the polarization for each angle. Calling $I_{p}$ the total intensity of light that makes it through the polarizer:
\begin{gather*}
    I_{p} = 4\int_{0}^{\pi/2}\frac{I_{tot}\sin(t)}{2\pi}dt = -\frac{2}{\pi}I_{tot}\cos(t)\Big|_{0}^{\pi/2} = \frac{2}{\pi}I_{tot} 
\end{gather*}
Or, the proportion of the light let through is:
\begin{gather*}
    \frac{I_{p}}{I_{tot}} = \frac{2}{\pi}
\end{gather*}
What happens if we introduce two polarizers though. From observation we know that light \emph{will} pass through if they are non-perpendicular. To see the proportion of light that passes through both, we need to see the projection of the orientation of the light from one polarizer onto the second. This will be the dot product between the two direction vectors for the polarizers. For example, if one polarizer points in the direction of the $y$-axis, and the second polarizer is rotated an angle $\theta$ counter clockwise, the projection of the vectors that pass through oriented to the $y$-axis onto this new polarizer are $I\cos(\theta)$. Thus, passing between two polarizers, the proportion of light that makes it through should be:
\begin{gather*}
    \frac{I_{2p}}{I_{tot}} = \frac{2}{\pi}\cos(\theta)
\end{gather*}
Where $\theta$ is the angle between them. Note, this means that if the polarizers are orthogonal in orientation $\theta = \pi/2$ and this collapses to zero. Interestingly, if you introduce a polarizer in the middle that is not parallel to either of the two perpendicular polarizers, light will pass through. It's a series of projections, that strip out each of the perpendicular components before doing the next projection. Thus we see components that might act in the direction of one of the previous polarizations being retintroduced, which is a wild phenomenon. Consider three polarizers, two orthogonal with a third in between that is exactly in the middle of the two. The initial intensity gets scaled by a factor of $2 / 2\pi$, then it gets scaled by two factors of $\cos(\theta)$ where $\theta = \pi/4$:
\begin{gather*}
    \frac{I_{3p}}{I_{tot}} = \frac{2}{\pi}\cos^{2}(\pi / 4) = \frac{2}{\pi}\frac{2}{4} = \frac{1}{\pi}
\end{gather*}
Or exactly half of the original intesity let through just one polarizer. If there are $n$ polarizers evenly distributed between the two:
\begin{gather*}
    \frac{I_{np}}{I_{tot}} = \frac{2}{\pi}\cos^{n}\left(\frac{\pi}{2n}\right) 
\end{gather*}
In the limit that $n \rightarrow \infty$, we see:
\begin{gather*}
    \frac{\pi}{2n} \rightarrow 0\\
    \cos^{n}\left(\frac{\pi}{2n}\right) \rightarrow 1
\end{gather*}
Or it's like it's only going through a single polarizer, which is an interesting result.
\end{document}